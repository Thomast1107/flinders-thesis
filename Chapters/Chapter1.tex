% Chapter Template

\chapter{Project Proposal} \label{chapter:firstchapter} % Main chapter title

\label{ChapterX} % Change X to a consecutive number; for referencing this chapter elsewhere, use \ref{ChapterX}

%----------------------------------------------------------------------------------------
%	SECTION 1 
%----------------------------------------------------------------------------------------


% It is a good idea to have each sentence on a separate line, so that if you get feedback or changes from someone else
% the diffs will be much easier to manage




%-----------------------------------
%	SUBSECTION 1
%-----------------------------------
\subsection{Project Focus}


Developed countries have vital sign monitoring device which are more expensive and they are less affordable for backward communities. 
This project aims in providing inexpensive monitoring device platform with the help of PCI controller. This platform will be helpful for the hospitals and researchers during power failure or shutdown.
Pulse oximetry is a cheap and non-invasive method of measurement which can be carried ouy by counting the pulse rate. In order to reduce the expense fingertip sensor can be easily obtained from the local drug shop around dollar \$25USD.\cite{HACKADAY}
Backward areas and disaster prone areas may lack energy supply for the normal working of the medical device. This can be rectified by using the solar panels in the backside of the device. The platform also provides a display unit of 1/2inch 800x450.

The main focus of the project is to reduce the private competitive mentality and to provide a cheap platform for the startups.
 

%-----------------------------------
%	SUBSECTION 2
%-----------------------------------

\subsection{Project Problems}

Every project implementation process has got its own pros and cons. Pros includes many positive innovations for this device platform.
The main problem or cons for this project is the cost of certification. 
This process of certification is a main problem in the implementation and the determination of the cost of that device. 
This is the only reason for the hike in the price of the medical devices in the current market.

The second problem which can be pointed out is the implementation of the solar panel in the medical device.
 As per the current market it  is expensive to implement the solar panel in the medical device.
This may result in the increased expense while building the platform.
 But it can also be achieved in a very low cost.

Selection of sensors is also a hard task in the pulse oximetry as the different sensors have got different input ports. 
This has to be checked before interfacing the hardware port.

Temperature comphensation should also be considered. 

%----------------------------------------------------------------------------------------
%	SUBSECTION 3
%----------------------------------------------------------------------------------------

\subsection{Research Questions}


%----------------------------------------------------------------------------------------
%	SUBSUBSECTION 3
%----------------------------------------------------------------------------------------

\subsubsection{How will people accept this project idea?}

In a recent study which carried out in OXFORD Academic by \cite{schermer2009pulse,}  shows that  pulse oximetry added as a much valuable asssesment method in every family. 
This is also used to explore the association between the spo2 and the marker for COPD severity when it os used by family physicians.  
Among 88 patients who were suffering from deterioratng COPD 22 percent of them showed spo2 which was $\geq$ 92 percent .  
When it was checked for 207 patients with stable COPD ,6.3 percent showed spo2 equal and above 92 percent .
This study showed how pulse oximetry become a major part of the health assessment.  

%----------------------------------------------------------------------------------------
%	SUBSUBSECTION 3
%----------------------------------------------------------------------------------------

\subsubsection{What makes this project different from other competitors?}

There are many features which make this project different from others :
1. The cost of the medical device platform is much less
2. Powerconsumption is much less as it is compared with othe controller units.
3. Size of the medical device paltform is less which makes it more handy for use.
4. This provides a medical device platform for the startups to implement their innovative ideas. This can be achieved in low cost.
5. This can make a big change in the medical field among other competitors.
6. This medical platform focus mainly on the backward and disaster prone areas and people.
7. They use very data level for the transmission of data.   

%----------------------------------------------------------------------------------------
%	SUBSECTION 4
%----------------------------------------------------------------------------------------

\subsection{Background Survey}

Paper 1 :PULSE OXIMETER IS A LOT OF WORK \cite{HACKADAY}
Author  : AL Williams  

Date: March 11 2017 
 
Pulse oximetry which was build with the help of Raspberry pi controller. 
They used MAX30100, 30101 ,30102  spo2 probe sensor for attaining the signal. 
The detector traditiobnally consist of two LED one of red and other of IR. 
The data acquired by the probe was stored in a buffer. 
Using the help of code the DC components were removed from the output. 
They took more time for processing the information. 

Paper 2 :  Microcontroller Based Pulse Oximeter for Undergraduate Capstone Design \cite{tamayo2010microcontroller}

Author  :Michael Tamayo, Andrew Westover, Ying Sun, PhD.

Date: 

Pulseoximetry which is used as a non-invasive method for measuring the blood oxygen saturation level. 
They used PIC 18F452  mirocontroller. The whole circuit consist of a plethysmograph, DAC, transistor network and the screen for display. 
The coding were written in C++.
They acquired the signal outputs after a complex operatons.  

