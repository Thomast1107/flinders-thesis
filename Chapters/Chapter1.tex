% Chapter Template

\chapter{Project Proposal}\label{chapter:firstchapter} % Main chapter title

\label{ChapterX} % Change X to a consecutive number; for referencing this chapter elsewhere, use \ref{ChapterX}

%----------------------------------------------------------------------------------------
%	SECTION 1 
%----------------------------------------------------------------------------------------


% It is a good idea to have each sentence on a separate line, so that if you get feedback or changes from someone else
% the diffs will be much easier to manage




%-----------------------------------
%	SUBSECTION 1
%-----------------------------------
\subsection{Project Focus}


Developed countries have vital sign monitoring device which are more expensive and they are less affordable for backward communities. This project aims in providing inexpensive monitoring device platform with the help of PCI controller. This platform will be helpful for the hospitals and researchers during power failure or shutdown. (Chapter \ref{chapter:firstchapter}).Pulse oximetry is a cheap and noninvasive method of measurement which can be carried ouy by counting the pulse rate. In order to reduce the expense fingertip sensor can be easily obtained from the local drug shop around dollar 25USD. (Section \ref{sec:firstsection}).Backward areas and disaster prone areas may lack energy supply for the normal working of the medical device. This can be rectified by using the solar panels in the backside of the device. The platform also provides a display unit of 1/2inch 800x450.

The main focus of the project is to reduce the private competitive mentality and to provide a cheap platform for the startups.
 

%-----------------------------------
%	SUBSECTION 2
%-----------------------------------

\subsection{Subsection 2}
Morbi rutrum odio eget arcu adipiscing sodales.
Aenean et purus a est pulvinar pellentesque.
 Cras in elit neque, quis varius elit.
 Phasellus fringilla, nibh eu tempus venenatis, dolor elit posuere quam, quis adipiscing urna leo nec orci.
 Sed nec nulla auctor odio aliquet consequat.
 Ut nec nulla in ante ullamcorper aliquam at sed dolor.
 Phasellus fermentum magna in augue gravida cursus.
 Cras sed pretium lorem.
 Pellentesque eget ornare odio.
 Proin accumsan, massa viverra cursus pharetra, ipsum nisi lobortis velit, a malesuada dolor lorem eu neque.

%----------------------------------------------------------------------------------------
%	SECTION 2
%----------------------------------------------------------------------------------------

\section{Main Section 2}

Sed ullamcorper quam eu nisl interdum at interdum enim egestas.
 Aliquam placerat justo sed lectus lobortis ut porta nisl porttitor.
 Vestibulum mi dolor, lacinia molestie gravida at, tempus vitae ligula.
 Donec eget quam sapien, in viverra eros.
 Donec pellentesque justo a massa fringilla non vestibulum metus vestibulum.
 Vestibulum in orci quis felis tempor lacinia.
 Vivamus ornare ultrices facilisis.
 Ut hendrerit volutpat vulputate.
 Morbi condimentum venenatis augue, id porta ipsum vulputate in.
 Curabitur luctus tempus justo.
 Vestibulum risus lectus, adipiscing nec condimentum quis, condimentum nec nisl.
 Aliquam dictum sagittis velit sed iaculis.
 Morbi tristique augue sit amet nulla pulvinar id facilisis ligula mollis.
 Nam elit libero, tincidunt ut aliquam at, molestie in quam.
 Aenean rhoncus vehicula hendrerit.
